\documentclass[a4paper]{article}
\usepackage{enteteTD}

\usepackage[T1]{fontenc}
\usepackage[utf8]{inputenc}
\usepackage[dvips]{graphicx}
\usepackage{amsmath}
\usepackage{amssymb}
\usepackage[francais]{babel}
\usepackage{euscript}
\usepackage{boxedminipage}
\usepackage{theorem}
\theoremstyle{break}
\usepackage{multirow}
\usepackage{xspace}
\usepackage{url}
\usepackage{epsfig}
\usepackage[ruled,vlined]{algorithm2e}

%\usepackage[latin1]{inputenc}
%\usepackage{epsfig}

\setlength{\textheight}{23.5cm}
\setlength{\topmargin}{-1cm}
\setlength{\textwidth}{155mm}
\setlength{\oddsidemargin}{2mm}

\begin{document}

%------------------- TITRE -----------------------------------------
\date{\today}
\TDHead{Pierre RAMET}{Licence SIL}{\large Android C++}  
%-------------------------------------------------------------------

Ce document est basé sur l'article "Android C++" du numéro 154 de Linux Magazine
France par Y.~Bailly.

\medskip
\hrule
\medskip

\section{Configuration}

    $\blacktriangleright$ Kits de développement

Pour pouvoir développer une application en C/C++ pour la plateforme
Android, il est nécessaire d'installer le NDK (kit natif) ainsi que le
SDK (kit standard). 
Au département, ces environnements de développement ont été installés
dans {\tt /info-nfs/opt/Android-sdk/}.

Il vous faut cependant positionner les variables d'environnement
suivantes, par exemple :

\begin{verbatim}
export ANDROID_SDK=/Users/ramet/Work/android/android-sdk-macosx
export PATH=$ANDROID_SDK/tools:$ANDROID_SDK/platform-tools:$PATH
export ANDROID_NDK=/Users/ramet/Work/android/android-ndk-r8c
export PATH=$ANDROID_NDK:$PATH
\end{verbatim}

    $\blacktriangleright$ Périphérique virtuel

Pour tester le code que nous allons écrire, il est nécessaire de
disposer d'un périphérique virtuel ou physique.
Pour configurer un périphérique virtuel, utilisez la commande~:

\begin{verbatim}
android avd
\end{verbatim}

puis démarrez le périphérique. Les options par défaut sont
suffisantes, notez cependant le niveau d'API disponible 
({\tt Android 4.2 - API Level 17}, par exemple). 
Pour vérifier que le périphérique est bien reconnu, exécutez la commande~:

\begin{verbatim}
adb devices
\end{verbatim}

\section{Création d'un projet}

Il faut commencer par créer une arborescence initiale~:

\begin{verbatim}
mkdir mandel && cd mandel
mkdir -p jni res/values
\end{verbatim}

A la racine du projet, il faut créer un fichier 
"{\tt Android-Manifest.xml}":

\begin{verbatim}
<?xml version="1.0" encoding="utf-8"?>
<manifest xmlns:android="http://schemas.android.com/apk/res/android"
      package="com.example.mandel"
      android:versionCode="1"
      android:versionName="1.0">
    <uses-sdk android:minSdkVersion="9" />
    <application android:label="@string/app_name"
                 android:hasCode="false" android:debuggable="true">
        <activity android:name="android.app.NativeActivity"
                  android:label="@string/app_name"
                  android:configChanges="keyboard|keyboardHidden|orientation">
            <meta-data android:name="android.app.lib_name"
                    android:value="mandel" />
            <intent-filter>
                <action android:name="android.intent.action.MAIN" />
                <category android:name="android.intent.category.LAUNCHER" />
            </intent-filter>
        </activity>
    </application>
</manifest>
\end{verbatim}

ainsi qu'un fichier "{\tt default.properies}" contenant le niveau API
de la cible~:

\begin{verbatim}
target=android-17
\end{verbatim}

Dans le répertoire {\tt res/values}, il faut nommer l'application avec un fichier 
 "{\tt strings.xml}" contenant~:

\begin{verbatim}
<?xml version="1.0" encoding="utf-8"?>
<resources>
    <string name="app_name">Mandel</string>
</resources>
\end{verbatim}

Enfin, créez deux fichiers dans le répertoire {\tt jni}, 
le premier "{\tt Application.mk}"~:

\begin{verbatim}
APP_ABI := armeabi
APP_PLATFORM := android-10
\end{verbatim}

et le second "{\tt Android.mk}"~:

\begin{verbatim}
LOCAL_PATH := $(call my-dir)
include $(CLEAR_VARS)
LOCAL_MODULE    := mandel
LOCAL_SRC_FILES := mandel.cpp
LOCAL_LDLIBS    := -lm -llog -landroid
LOCAL_STATIC_LIBRARIES := android_native_app_glue
include $(BUILD_SHARED_LIBRARY)
\end{verbatim}

\section{Le code}

Les sources du projet sont disponibles dans le répertoire {\tt
  /info-nfs/users/pramet/mandel\_android.tar.gz}.

La partie concernant l'affichage de l'ensemble de MandelBrot se trouve
dans la fonction {\tt draw\_frame()}. 

La gestion des évènements est
traitée dans la fonction {\tt engine\_handle\_cmd()}.

La mise en place de l'application débute avec la fonction {\tt android\_main()}.

\newpage

\section{Compiler et déployer}

Pour compiler le projet~:

\begin{verbatim}
android update project -p . -s
ndk-build -B V=1
ant debug
\end{verbatim}

Il ne reste plus qu'à envoyer le paquetage sur le périphérique
virtuel~:

\begin{verbatim}
adb -e install bin/NativeActivity-debug.apk
\end{verbatim}

ou physique~:

\begin{verbatim}
adb -d install bin/NativeActivity-debug.apk
\end{verbatim}

Pour visualiser la sortie "journal" de l'application (cf la macro {\tt LOGW})~:

\begin{verbatim}
adb -e logcat
\end{verbatim}

Pour désinstaller l'application~:

\begin{verbatim}
adb -e uninstall com.example.mandel
\end{verbatim}

\end{document}

